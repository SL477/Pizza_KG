% !TEX TS-program = pdflatex
% !TEX encoding = UTF-8 Unicode

% This is a simple template for a LaTeX document using the "article" class.
% See "book", "report", "letter" for other types of document.

\documentclass[11pt]{report} % use larger type; default would be 10pt

\usepackage[utf8]{inputenc} % set input encoding (not needed with XeLaTeX)

%%% Examples of Article customizations
% These packages are optional, depending whether you want the features they provide.
% See the LaTeX Companion or other references for full information.

%%% PAGE DIMENSIONS
\usepackage{geometry} % to change the page dimensions
\geometry{a4paper} % or letterpaper (US) or a5paper or....
% \geometry{margin=2in} % for example, change the margins to 2 inches all round
%   read geometry.pdf for detailed page layout information

\usepackage{graphicx} % support the \includegraphics command and options

% \usepackage[parfill]{parskip} % Activate to begin paragraphs with an empty line rather than an indent

%%% PACKAGES
\usepackage{booktabs} % for much better looking tables
\usepackage{array} % for better arrays (eg matrices) in maths
\usepackage{paralist} % very flexible & customisable lists (eg. enumerate/itemize, etc.)
\usepackage{verbatim} % adds environment for commenting out blocks of text & for better verbatim
\usepackage{subfig} % make it possible to include more than one captioned figure/table in a single float
% These packages are all incorporated in the memoir class to one degree or another...

%%% HEADERS & FOOTERS
\usepackage{fancyhdr} % This should be set AFTER setting up the page geometry
\pagestyle{fancy} % options: empty , plain , fancy
\renewcommand{\headrulewidth}{0pt} % customise the layout...
\lhead{}\chead{}\rhead{}
\lfoot{}\cfoot{\thepage}\rfoot{}

%%% SECTION TITLE APPEARANCE
\usepackage{sectsty}
\allsectionsfont{\sffamily\mdseries\upshape} % (See the fntguide.pdf for font help)
% (This matches ConTeXt defaults)

%%% ToC (table of contents) APPEARANCE
\usepackage[nottoc,notlof,notlot]{tocbibind} % Put the bibliography in the ToC
\usepackage[titles,subfigure]{tocloft} % Alter the style of the Table of Contents
\renewcommand{\cftsecfont}{\rmfamily\mdseries\upshape}
\renewcommand{\cftsecpagefont}{\rmfamily\mdseries\upshape} % No bold!

%%% END Article customizations

%Report: \part{}, \chapter{}, \section{}, \subsection{}, \subsubsection{}, \paragraph{}, \subparagraph{}.

\usepackage{csvsimple}
%\usepackage{pgfplotstable}
\usepackage{longtable} 
\usepackage{listings}
\usepackage[dvipsnames]{xcolor}
\usepackage{hyperref}

\lstset{
  basicstyle=\ttfamily,
  columns=fullflexible,
  frame=single,
  breaklines=true,
  postbreak=\mbox{\textcolor{red}{$\hookrightarrow$}\space},
}

%\makeatletter
%\csvset{
 % my column width/.style={after head=\csv@pretable\begin{longtable}{*{\csv@columncount}{p{#1}}}\csv@tablehead},
%}
%\makeatother

%%% The "real" document content comes below...

\title{Pizza Knowledge Graph}
\author{Thomas Fishwick}
%\date{} % Activate to display a given date or no date (if empty),
         % otherwise the current date is printed 


\begin{document}

\begin{minipage}{\textwidth}
	\maketitle

	\begin{abstract}
		abstract-text
	\end{abstract}
\end{minipage}

\tableofcontents

\chapter{Ontology Modelling}

For my knowledge graph I have chosen to seperate out the country, province, City, restaurant and menu items into different classes.
This is so that repeating data is generally only stored once and to enable us to model the hiarchy of the data.
For the pizzas I have modelled various different types of pizza and their ingredients.
In theory this should let us reaon the type of a pizza from the ingredients. I also added in generic types of pizza, e.g. vegetarian, meat.

I put in the various relationships between the data, with their opposite types. Such as City has Restaurant and City Located in.
I also put in various data properties which did not need to go into the class hiarchy, such as the latitude and longditude of the restaurant.

\chapter{Tabular Data to Knowledge Graph}

For getting the state and country data no entity resolution was needed.
For the cities, some more creativity was needed to match some of the cities up after the bulk of them were matched automatically from using SPARQL to get lists of cities from DBPedia.

To resolve the entities I built a map with words to look for in the menu name back to the ontology URL. I did the same thing for the ingredients, but split them up as well.

\chapter{SPARQL \& Reasoning}
\section{Restaurants Selling Pizza Bianca}
The following SPARQL query returns the table \url{https://github.com/SL477/Pizza_KG/blob/main/sparqlAndReasoning/pizzaBiancaRestaurants.csv} with the restaurants which serve Bianca pizzas.

\lstinputlisting[caption={../sparqlAndReasoning/pizza\_bianca\_restaurants\_short.sparql}]{../sparqlAndReasoning/pizza_bianca_restaurants_short.sparql}

%\csvautobooklongtable[respect all]{../sparqlAndReasoning/pizzaBiancaRestaurants.csv}

\section{Average Price of a margherita pizza}

The following SPARQL query returns the average price of a Margherita pizza.

\lstinputlisting[caption={../sparqlAndReasoning/average\_price\_margherita.sparql}]{../sparqlAndReasoning/average_price_margherita.sparql}

The average price is \$12.05

\section{Number of restaurants}

The following SPARQL query gets the number of restaurants per state and city and saves them to \url{https://github.com/SL477/Pizza_KG/blob/main/sparqlAndReasoning/number_restaurants.csv}.

\lstinputlisting[caption={../sparqlAndReasoning/number\_restaurants.sparql}]{../sparqlAndReasoning/number_restaurants.sparql}

\section{Restaurants missing postcode}

The following SPARQL query finds the restaurants missing their postcode and saves them to \url{https://github.com/SL477/Pizza_KG/blob/main/sparqlAndReasoning/restaurants_missing_postcode.csv}.

\lstinputlisting[caption={../sparqlAndReasoning/number\_restaurants\_missing\_postcode.sparql}]{../sparqlAndReasoning/restaurants_missing_postcode.sparql}

\end{document}
